\documentclass[compress,usepdftitle=false,xcolor=pdftex,svgnames,table,aspectratio=169, 14pt]{beamer}
\usepackage[utf8]{inputenc}

% Notes
% \usepackage{pgfpages}
% \setbeameroption{show notes on second screen=right}

%\definecolor{bluegreen}{RGB}{3, 166, 155}
%\definecolor{pitchblack}{RGB}{0, 0, 0}
%\definecolor{lightbeige}{RGB}{255, 251, 241}
%\definecolor{mediumgray}{RGB}{183, 183, 183}

%\setbeamercolor{background canvas}{bg=pitchblack}
\setbeamercolor{normal text}{fg=black}
\setbeamercolor{frametitle}{bg=LightGrey, fg=black}
\setbeamercolor{title}{fg=black, bg=white}
\definecolor{blockcolor}{RGB}{250, 250, 200}
\setbeamercolor{block}{fg=black, bg=blockcolor}
\setbeamercolor{sticky}{fg=black, bg=blockcolor}
%\usetheme{plain}
%\useoutertheme[subsection=false]{miniframes}
\usefonttheme{professionalfonts}
%%\setbeamercolor{section in head/foot}{fg=white, bg=black}
%% \setbeamercolor{mini frame}{fg=white, bg=black}
\usebeamercolor[fg]{title in head/foot}
\usefonttheme[onlylarge]{structurebold}
\setbeamerfont*{frametitle}{size=\normalsize,series=\bfseries}
\setbeamertemplate{navigation symbols}{}
%%\setbeamertemplate{headline}[miniframes theme]{}
%\setbeamertemplate{footline}[default]
%\defbeamertemplate{footline}{centered page number}
%{%
%  \hspace*{\fill}%
%  \usebeamercolor[fg]{page number in head/foot}%
%  \usebeamerfont{page number in head/foot}%
%  \insertpagenumber\,/\,\insertpresentationendpage%
%  \hspace*{\fill}\vskip2pt%
%}
%\setbeamertemplate{footline}[centered page number]
\setbeamertemplate{footline}[frame number]
\setbeamertemplate{bibliography entry title}{}
\setbeamertemplate{bibliography entry location}{}
\setbeamertemplate{bibliography entry note}{}

%\setbeamercolor{background canvas}{bg=LightGrey}

\makeatletter
\setbeamertemplate{title page}[default][left,colsep=-4bp,rounded=true]
\makeatother
\setbeamerfont{frametitle}{size=\large}
\setbeamerfont{title}{size=\Large}
\setbeamerfont{subtitle}{size=\large}
\setbeamerfont{author}{size=\small}
\setbeamerfont{institute}{size=\small}

\setbeamercolor*{block title alerted}{fg=black, bg=red!40}
\setbeamercolor*{block body alerted}{fg=black, bg=red!25}
\setbeamerfont{alerted text}{series=\bfseries}
\setbeamercolor{alerted text}{fg=blue, bg=white}

\setbeamercolor*{block title example}{fg=blue!50, bg=blue!10}
\setbeamercolor*{block body example}{fg= blue, bg=blue!5}

\newcommand\mailto[1]{\href{mailto: #1}{#1}}

\newcommand\redbf[1]{{\color{red}\textbf{#1}}}

\newcommand\itm{\color{gray}\scriptsize\raise1.25pt\hbox{\textbullet}}

\setbeamertemplate{itemize item}{\scriptsize\raise1.25pt\hbox{\textbullet}}
\setbeamertemplate{itemize subitem}{\tiny\raise1.5pt\hbox{\textbf{--}}}
\setbeamertemplate{itemize subsubitem}{\tiny\raise1.5pt\hbox{\textbf{--}}}
\setbeamertemplate{enumerate item}{\insertenumlabel.}
\setbeamertemplate{enumerate subitem}{\insertenumlabel.\insertsubenumlabel}
\setbeamertemplate{enumerate subsubitem}{\insertenumlabel.\insertsubenumlabel.\insertsubsubenumlabel}
\setbeamertemplate{enumerate mini template}{\insertenumlabel}
\setbeamercolor{itemize item}{fg=gray}
\setbeamercolor{itemize subitem}{fg=gray}
\setbeamercolor{itemize subsubitem}{fg=gray}

\usepackage{minted}
% \usepackage{listings}
%\input{lstcoq}
\usepackage{tikz}
  \usetikzlibrary{calc,decorations.pathreplacing,
    decorations.pathmorphing,
    decorations.markings,
    shapes, fit,
    arrows.meta
    }
  \usetikzlibrary{cd}
  \usetikzlibrary{positioning}
  \usetikzlibrary{arrows}

\tikzset{
    arrow at end/.style={
        decorate,decoration={
            markings,
            mark=at position .999 with{
                \arrow{#1};
            }
        }
    }
}

% alternative stealth arrow
\pgfarrowsdeclare{stealthnew}{stealthnew}
{
  \ifdim\pgfgetarrowoptions{stealthnew}=-1pt%
    \pgfutil@tempdima=0.28pt%
    \pgfutil@tempdimb=\pgflinewidth%
    \ifdim\pgfinnerlinewidth>0pt%
      \pgfmathsetlength\pgfutil@tempdimb{.6\pgflinewidth-.4*\pgfinnerlinewidth}%
    \fi%
    \advance\pgfutil@tempdima by.3\pgfutil@tempdimb%
  \else%
    \pgfutil@tempdima=\pgfgetarrowoptions{stealthnew}%
    \divide\pgfutil@tempdima by 8%
  \fi%
  \pgfarrowsleftextend{+-3\pgfutil@tempdima}
  \pgfarrowsrightextend{+5\pgfutil@tempdima}
}
{
  \ifdim\pgfgetarrowoptions{stealthnew}=-1pt%
    \pgfutil@tempdima=0.28pt%
    \pgfutil@tempdimb=\pgflinewidth%
    \ifdim\pgfinnerlinewidth>0pt%
      \pgfmathsetlength\pgfutil@tempdimb{.6\pgflinewidth-.4*\pgfinnerlinewidth}%
    \fi%
    \advance\pgfutil@tempdima by.3\pgfutil@tempdimb%
  \else%
    \pgfutil@tempdima=\pgfgetarrowoptions{stealthnew}%
    \divide\pgfutil@tempdima by 8%
    \pgfsetlinewidth{0bp}%
  \fi%
  \pgfpathmoveto{\pgfqpoint{5\pgfutil@tempdima}{0pt}}
  \pgfpathlineto{\pgfqpoint{-3\pgfutil@tempdima}{4\pgfutil@tempdima}}
  \pgfpathlineto{\pgfpointorigin}
  \pgfpathlineto{\pgfqpoint{-3\pgfutil@tempdima}{-4\pgfutil@tempdima}}
  \pgfusepathqfill
}


\usepackage[skins]{tcolorbox}

\newtcolorbox{sticky}[1][]
{%
  enhanced,
  center upper,
  fontupper=\strut,
  drop fuzzy shadow southeast,
  boxrule=0pt,
  sharp corners,
  colframe=yellow!80!black,
  colback=yellow!10,
  #1
}

\newtcolorbox{infobox}[1][]
{%
  enhanced,
  %center upper,
  %drop fuzzy shadow southeast,
  boxrule=0.4pt,
  sharp corners,
  colframe=red!80!black,
  colback=red!10,
  #1
}

\newtcolorbox{bluebox}[1][]
{%
  enhanced,
  %center upper,
  %drop fuzzy shadow southeast,
  boxrule=0.4pt,
  sharp corners,
  colframe=blue!80!black,
  colback=blue!10,
  #1
}

\newtcolorbox{greenbox}[1]
{%
  enhanced,
  %center upper,
  drop fuzzy shadow southeast,
  boxrule=0pt,
  sharp corners,
  colframe=green!80!black,
  colback=green!10,
  #1
}
\newtcolorbox{redbox}[1][]
{%
  enhanced,
  % center upper,
   drop fuzzy shadow southeast,
  boxrule=0pt,
  sharp corners,
  colframe=red!80!black,
  colback=red!10,
  #1
}
\usepackage{microtype}
\newsavebox\CBox
\newcommand<>*\textBF[1]{\sbox\CBox{#1}\resizebox{\wd\CBox}{\ht\CBox}{\textbf#2{#1}}}
\usepackage{tabularx}
\usepackage[framemethod=TikZ]{mdframed}

\newmdenv[
  topline=false,
  bottomline=false,
  rightline=false,
  skipabove=\topsep,
  skipbelow=\topsep,
  linewidth=1pt,
  linecolor=cyan
]{titlerule}

\setbeamertemplate{frametitle}{%
  \begin{titlerule}
    \usebeamerfont{frametitle}\insertframetitle\strut%
    \vskip-.2\baselineskip%
    % \leaders\vrule width \paperwidth\vskip0.4pt%
    \vskip0pt%
    \nointerlineskip
  \end{titlerule}
}


\usepackage{bm}

% \titlegraphic{%
%   \begin{picture}(382,0)
%     {
%       \put(150,22){\includegraphics[width=3cm]{figures/imperial_logo.pdf}}
%       \put(260,10){\includegraphics[width=2.5cm]{figures/Uok_Logo_RGB294.jpg}}
%     }
%   \end{picture}
%   }


\title{Act II}
\subtitle{Smol-Zooid: multiparty with shallower embedding}
\author{}
%\institute{}
\date{}
% \date{%\put(0,-60){
%   17-05-2021
% %}
% }
\usepackage{tipa} % for phonetic characters
\usepackage{graphicx}
\usepackage{ulem} % for strikethrough text with \sout{}
\usepackage{mathtools} % \Coloneqq

% Required packages

% \usepackage{proof}
\usepackage{mathpartir}
% \usepackage[svgnames]{xcolor}

% symbols
\usepackage{amsmath}
\usepackage{amssymb}
% \usepackage{dsfont}
% \usepackage{stmaryrd}


% \usepackage[normalem]{ulem} % for strikethrough with \sout{}
\usepackage{cancel}

% useful packages

\usepackage{xspace}
\usepackage{fancyvrb}
\VerbatimFootnotes % to enable the use of verbatim in footnotes, it may conflict with some packages

% syntax highlighting

% \usepackage[skins,listings]{tcolorbox}
\usepackage{listings}
\definecolor{dkblue}{rgb}{0,0.1,0.5}
\definecolor{lightblue}{rgb}{0,0.5,0.5}
\definecolor{dkgreen}{rgb}{0,0.4,0}
\definecolor{dk2green}{rgb}{0.4,0,0}
\definecolor{dkviolet}{rgb}{0.6,0,0.8}
\definecolor{mantra}{rgb}{0.2,0.6,0.2}
\definecolor{gotcha}{rgb}{0.8,0.2,0}
\definecolor{ocre}{RGB}{243,102,25} %borrowed to Orange Book

\def\lstlanguagefiles{defManSSR.tex}
\lstset{language=SSR}

%%% Some macros

% Miscelaneous typesetting
\newcommand{\code}[1]{\texttt{#1}} % typeset with mono spaced font
\newcommand{\rulename}[1]{\DefTirName{#1}}
\newcommand{\fnm}[1]{\ensuremath{\text{#1}}} % the name of a function
\newcommand{\dbj}{de Bruijn\xspace}

%% Color management

% \newcommand{\NOCOLOR} % define to remove color from the syntax

\ifdefined\NOCOLOR
  \newcommand{\withcolor}[2]{#2} % no color selsected
\else
  % sets and restores the color
  \newcommand{\withcolor}[2]{\colorlet{currbkp}{.}\color{#1}{#2}\color{currbkp}}
\fi

% to cancel terms in red instead of black
\ifdefined\NOCOLOR
\else
  \renewcommand{\CancelColor}{\color{red}}
\fi

% Macros for BNF grammars

\newcommand{\bnfas}{\mathrel{::=}}
\newcommand{\bnfalt}{\mathrel{\mid}}

% Optional color definitions

% no color
% \newcommand{\colorch}{black} % color for channel vars
% \newcommand{\colorex}{black} % color for expresion vars
% \newcommand{\colorse}{black} % color for session vars
% \newcommand{\colorlbl}{black} % color for labels
% \newcommand{\colorproc}{black} % color for processes
% \newcommand{\colorexp}{black} % color for expressions
% \newcommand{\colorte}{black} % color for the types of expressions
% \newcommand{\colortp}{black} % color for the types of processes

% some colors
\newcommand{\colorch}{DarkGreen} % color for channel vars
\newcommand{\colorex}{Tomato} % color for expresion vars
\newcommand{\colorse}{Teal} % color for session vars
\newcommand{\colorlbl}{Indigo} % color for labels
\newcommand{\colorproc}{Maroon} % color for processes
\newcommand{\colorexp}{Tomato} % color for expressions
\newcommand{\colorte}{DarkOrchid} % color for the types of expressions
\newcommand{\colortp}{NavyBlue} % color for the types of processes

%% Macros for kinds of vars
% these are to control the rendering of the different kinds of variables
\newcommand{\vch}[1]{\withcolor{\colorch}{#1}} % variables for channels
\newcommand{\vex}[1]{\withcolor{\colorex}{#1}} % variables for expressions
\newcommand{\vse}[1]{\withcolor{\colorse}{#1}} % variables for sessions

%% Macros for labels

\newcommand{\dlbl}[1]{\withcolor{\colorlbl}{#1}} % just to define possible use of colours
\newcommand{\lblleft}{\dlbl{\code{l}}}
\newcommand{\lblright}{\dlbl{\code{r}}}
% \newcommand{\lblleft}{\dlbl{\code{left}}}
% \newcommand{\lblright}{\dlbl{\code{right}}}

%% Macros for processes

\newcommand{\openop}[2]{\ensuremath{{#1}^{#2}}} % term name
\newcommand{\closeop}[2]{^{\backslash #2}\ensuremath{{#1}}} % term name

\newcommand{\dproc}[1]{\withcolor{\colorproc}{#1}} % just to define possible use of colours

\newcommand{\preq}[3]{\dproc{\code{request}\ \vse{#1}\,(\vch{#2}). \dproc{#3}}}
\newcommand{\pacc}[3]{\dproc{\code{accept}\ \vse{#1}\,(\vch{#2}). \dproc{#3}}}
\newcommand{\psend}[3]{\dproc{\vch{#1}\,![\dexp{#2}];\,\dproc{#3}}}
\newcommand{\precv}[3]{\dproc{\vch{#1}\,?(\vex{#2}). \dproc{#3}}}
\newcommand{\psel}[3]{\dproc{\vch{#1}\triangleleft\dlbl{#2};\dproc{#3}}}
\newcommand{\pbran}[3]
  {\dproc{\vch{#1}\triangleright\{\lblleft:\dproc{#2}[\!]\lblright:\dproc{#3}\}}}
\newcommand{\pthrow}[3]{\dproc{\code{throw}\ \vch{#1}\,[\vch{#2}];\dproc{#3}}}
\newcommand{\pcatch}[3]{\dproc{\code{catch}\ \vch{#1}\,(\vch{#2}). \dproc{#3}}}
\newcommand{\pif}[3]
  {\dproc{\code{if}\ \dexp{#1}\ \code{then}\ \dproc{#2}\ \code{else}\ \dproc{#3}}}
\newcommand{\ppar}[2]{\dproc{#1}\mathrel{|}\dproc{#2}}
\newcommand{\pinact}{\dproc{\code{inact}}}
\newcommand{\phide}[3]{\dproc{\nu_{#1}\,({#2}).\dproc{#3}}}
\newcommand{\hideboth}{\{n,c\}}
\newcommand{\phidenm}[2]{\phide{n} {\vse{#1}} {#2}}
\newcommand{\phidech}[2]{\phide{c} {\vch{#1}} {#2}}
\newcommand{\pnu}[1]{\ensuremath{{\dproc{\nu.{#1}}}}}

%% Macros for expressions

\newcommand{\dexp}[1]{\withcolor{\colorexp}{#1}} % just to define possible use of colours

\newcommand{\etrue}{\dexp{\code{true}}}
\newcommand{\efalse}{\dexp{\code{false}}}
\newcommand{\elam}[1]{\dexp{\lambda.}\,#1}
\newcommand{\eapp}{\,}
\newcommand{\ezero}{\dexp{0}}
\newcommand{\eone}{\dexp{1}}
\newcommand{\etwo}{\dexp{2}}

\newcommand{\ett}{\ensuremath{\dexp{\texttt{tt}}}}
\newcommand{\eff}{\ensuremath{\dexp{\texttt{ff}}}}
\newcommand{\eunit}{\ensuremath{\dexp{()}}}

%% Macros for types

%\newcommand{\dual}[1]{\bar{#1}}
\newcommand{\dual}[1]{\overline{#1}}

% expressions
\newcommand{\dte}[1]{\withcolor{\colorte}{#1}} % just to define possible use of colours

\newcommand{\tbool}{\dte{\code{bool}}}
\newcommand{\tendp}[1]{\dte{\langle\dtp{#1},\dtp{\dual{#1}}\rangle}}

% processes
\newcommand{\dtp}[1]{\withcolor{\colortp}{#1}} % just to define possible use of colours

\newcommand{\tsende}[2]{\dtp{![\dte{#1}];{#2}}}
\newcommand{\tsendp}[2]{\dtp{![{#1}];{#2}}}
\newcommand{\trecve}[2]{\dtp{?[\dte{#1}];{#2}}}
\newcommand{\trecvp}[2]{\dtp{?[{#1}];{#2}}}
\newcommand{\toffer}[2]{\dtp{\&\{\lblleft:{#1}, \lblright:{#2}\}}}
\newcommand{\ttake}[2]{\dtp{\oplus\{\lblleft:{#1}, \lblright:{#2}\}}}
\newcommand{\tend}{\dtp{\code{end}}}
\newcommand{\tbot}{\dtp{\bot}}

%% Macros for judgments and such

\newcommand{\scon}{\equiv}
\newcommand{\sconalpha}{\equiv_\alpha}
\newcommand{\freenames}[1]{\fnm{fn}({#1})}
\newcommand{\stepsto}{\longrightarrow} % process steps to
\newcommand{\evalsto}{\downarrow} % expression evaluates to
\newcommand{\subst}[2]{[{#1}/{#2}]} % substitution
\newcommand{\esubst}[2]{[\dexp{#1}/\dexp{#2}]} % substitution for expressions

\newcommand{\oft}{\mathrel{:}} % is of type
\newcommand{\oftj}{\mathrel{\triangleright}} % is of type in a process judgment
\newcommand{\oftc}{\mathrel{:}} % is of type in a process judgment
\newcommand{\ofte}[3]{\dexp{#1}\vdash\dexp{#2}\oftc\dtp{#3}}
\newcommand{\oftp}[3]{\dexp{#1}\vdash\dproc{#2}\oftj\dproc{#3}}

\newcommand{\completed}[1]{\dproc{#1}\ \text{completed}}

% Macros for contexts and typings

\newcommand{\sorting}[1]{\dexp{#1}}
\newcommand{\typing}[1]{\dproc{#1}}
\newcommand{\emptyctx}{\ensuremath{\cdot}}
\newcommand{\typc}[3]{\dproc{#1},\vch{#2}\oft\dtp{#3}} % add to a typing
\newcommand{\ctxc}[3]{\dexp{#1},\vex{#2}\oft\dte{#3}} % add to a context

\newcommand{\typl}[3]{\dexp{\Gamma}(\vex{#2})=\dte{#3}} % lookup typ
\newcommand{\ctxl}[3]{\dexp{#1}(\vex{#2})=\dte{#3}} % lookup context

\newcommand{\join}[2]{\dproc{#1}\circ\dproc{#2}}
\newcommand{\comp}[2]{\dproc{#1}\asymp\dproc{#2}}

\newcommand{\dom}[1]{\ensuremath{\fnm{dom}({#1})}}
\newcommand{\doms}[1]{\ensuremath{\fnm{dom}(\sorting{#1})}} % domain of a sorting
\newcommand{\domt}[1]{\ensuremath{\fnm{dom}(\typing{#1})}} % domain of a typing
\newcommand{\compatible}[2]{\ensuremath{\typing{#1}\mathrel{\asymp}\typing{#2}}}
\newcommand{\compose}[2]{\ensuremath{\typing{#1}\mathrel{\circ}\typing{#2}}}

\newcommand{\fv}[1]{\ensuremath{\fnm{fv}({#1})}} % free variables
\newcommand{\fvp}[1]{\fv{\dproc{#1}}} % free variables in a proc


%% Macros for LN things

\newcommand{\atoms}{\ensuremath{\mathbb{A}}}
\newcommand{\atomset}[1]{\atoms_{\code{#1}}}
\newcommand{\atomsca}{\atomset{CA}}
\newcommand{\atomsna}{\atomset{NA}}
\newcommand{\atomska}{\atomset{KA}}
\newcommand{\atomsea}{\atomset{EA}}

% open/close process with expression
\newcommand{\oppe}[2]{\dproc{#1}^{\dexp{#2}}}
\newcommand{\clpe}[2]{^{\backslash\dexp{#2}}\dproc{#1}}

% open/close process with channel
\newcommand{\oppk}[2]{\dproc{#1}^{\vch{#2}}}
\newcommand{\clpk}[2]{^{\backslash\vse{#2}}\dproc{#1}}

% open/close process with session (name)
\newcommand{\oppn}[2]{\dproc{#1}^{\vse{#2}}}
\newcommand{\clpn}[2]{^{\backslash\vse{#2}}\dproc{#1}}

% co finite quantifications

\newcommand{\cofin}[2]{\forall {#1}\notin {#2},\ }
\newcommand{\cofine}[2]{\cofin{\vex{#1}} {#2}}
\newcommand{\cofink}[2]{\cofin{\vch{#1}} {#2}}
\newcommand{\cofinn}[2]{\cofin{\vse{#1}} {#2}}

% predicates

\newcommand{\pred}[2]{\ensuremath{\dproc{#1}\mathrel{\to}\dproc{#2}}} % process reduction

\newcommand{\lc}[1]{\fnm{lc}\ {#1}} % locally closed
\newcommand{\lcp}[1]{\lc{\dproc{#1}}}
\newcommand{\lce}[1]{\lc{\dexp{#1}}}

\newcommand{\body}[1]{\fnm{body}\ {#1}} % at most one free index (0)
\newcommand{\bodyp}[1]{\body{\dproc{#1}}}
\newcommand{\bodye}[1]{\body{\dexp{#1}}}

% some names

\newcommand{\nuscr}{$\nu$-Scr}

%%% Local Variables:
%%% mode: latex
%%% TeX-master: "draft"
%%% End:



\newcommand{\phoneticZooidA}{%
  \textprimstress{}zu\textlengthmark\textopeno\textsci{}d
}
\newcommand{\phoneticZooidB}{%
  \textprimstress{}z\textschwa%
  \raisebox{-.01cm}{\rotatebox[origin=c]{180}{\textscomega}}%
  \textopeno\textsci{}d
}
\newcommand{\phoneticZooidC}{%
  \textprimstress{}zu\textlengthmark\textsci{}d
}

\usepackage{soul}
\newcommand{\mathcolorbox}[2]{\colorbox{#1}{$\displaystyle #2$}}
\newcommand{\hlfancy}[2]{\sethlcolor{#1}\hl{#2}}

\setlength{\leftmargini}{10pt}
\begin{document}
\frame[plain]{\titlepage}
% \section{Introduction}

\endinput
\begin{frame}
  \vfill
  \centering
  %\begin{beamercolorbox}[sep=8pt,center,shadow=true,rounded=true]{block}
  \begin{sticky}
    \usebeamerfont{title}{\underline{Part I:}} \\
    {\normalfont\Large Introduction to Concurrent Programming in Java --
    Deadlocks} \par%
  \end{sticky}
  %\end{beamercolorbox}
  \vfill
\end{frame}

\begin{frame}
  \frametitle{About Today's Lecture}

  \begin{block}{So far in this course:}
      \begin{itemize}
        \item[{\color{DarkGreen}$\checkmark$}] Java Thread creation
        \item[{\color{DarkGreen}$\checkmark$}] Synchronisation: Java \mintinline{java}{synchronized}, monitors
      \end{itemize}
  \end{block}

  \begin{block}{Today}
      \begin{itemize}
        \item[{\color{DarkBlue}$\blacksquare$}] Deadlocks
        \item[{\color{DarkBlue}$\blacksquare$}] The Dining Philosophers Problem
      \end{itemize}
  \end{block}

  \begin{block}{\color{DarkGreen}Intended Learning Outcomes}
      \begin{itemize}
        \item[{\color{DarkGreen}$\blacksquare$}] The concept of \textbf{deadlock}
        \item[{\color{DarkGreen}$\blacksquare$}] Four necessary and sufficient conditions for deadlocks
        \item[{\color{DarkGreen}$\blacksquare$}] Identifying deadlocks in practice: blocked threads
      \end{itemize}
  \end{block}
\end{frame}

\begin{frame}
  \frametitle{Deadlock}
  \begin{tcolorbox}[colback=white,colframe=DarkGreen]
    \begin{tabularx}{\linewidth}{l l}
      \textbf{Concept:} & \alert{deadlock}: no further progress possible
      \\
      & there are four necessary \& sufficient conditions
      \\
      \textbf{In Practice:} & blocked threads
    \end{tabularx}
  \end{tcolorbox}

  \begin{minipage}{.4\textwidth}
    \hspace{10cm}
  \end{minipage}
  \begin{minipage}{.54\textwidth}
  \begin{alertblock}{Aim}
    \textbf{Deadlock-avoidance}: design and implement systems where deadlock cannot
    occur.
  \end{alertblock}
  \end{minipage}
\end{frame}

\begin{frame}
  \frametitle{Four necessary \& sufficient conditions for deadlocks}
  \begin{tcolorbox}[colback=white,colframe=DarkGreen]
    \begin{enumerate}
      \item \alert<2>{Serially reusable resources}
        \uncover<2->{
          \begin{itemize}
            \item[] the processes involved share resources which they use under mutual exclusion
          \end{itemize}
        }
      \item \alert<3>{Incremental acquisition}
        \uncover<3->{
          \begin{itemize}
            \item[] processes hold on to resources already allocated to them
              while waiting to acquire additional resources
          \end{itemize}
        }
        \item \alert<4>{No preemption}
        \uncover<4->{
          \begin{itemize}
            \item[] once acquired by a process, resources cannot be pre-empted
              (forcibly withdrawn) but are only released voluntarily
          \end{itemize}
        }
        \item \alert<5>{Wait-for cycle}
        \uncover<5->{
          \begin{itemize}
            \item[] a circular chain (or cycle) of processes exists such that
              each process holds a resource which its successor in the cycle is
              waiting to acquire
          \end{itemize}
        }
    \end{enumerate}
  \end{tcolorbox}
\end{frame}

\begin{frame}
  \frametitle{Wait-for cycle}

  \begin{tikzpicture}
    \node[draw, ultra thick, circle, minimum width=1cm] at (180:2cm) (A) { A };

    \node[draw, ultra thick, circle, minimum width=1cm] at (60:2cm) (B) { B };

    \node[draw, ultra thick, circle, minimum width=1cm] at (300:2cm) (C) { C };

    \node[draw, ultra thick, circle, minimum width=1cm, right=2cm of B] (Z) { Z };

    \node[left=.7cm of A] {\color{red}holds A, awaits B};
    \node[above=.4cm of B] {\color{red}holds B, awaits C};
    \node[below=.4cm of C] {\color{red}holds C, awaits A};
    \node[below=.4cm of Z] {\color{red}holds Z, awaits B};
    \draw[bend left=45, thick, ->,tips=proper] (A) to (B);
    \draw[bend left=45, thick, ->,tips=proper] (B) to (C);
    \draw[bend left=45, thick, ->,tips=proper] (C) to (A);
    \draw[thick, ->, tips=proper] (Z) -- (B);
  \end{tikzpicture}
\end{frame}

\begin{frame}
  \frametitle{Dining Philosophers (I)}

  \begin{columns}
    \begin{column}{.6\textwidth}
       Five philosophers sit around a circular table.
       \begin{itemize}
           \item Each philosopher spends
       his life alternately \alert{thinking} and \alert{eating}.
       % In the centre of the table is a large bowl of spaghetti.
           \item A philosopher needs \alert{two forks} to start eating.

          \item One fork is placed between each pair of philosophers and they agree that
            each will only use the fork to his immediate right and left.
       \end{itemize}
    \end{column}

    \begin{column}{.39\textwidth}
      \includegraphics[width=.9\textwidth]{figures/dining_philosophers.png}
      \hspace*{15pt}\hbox{\tiny Source:\thinspace{\itshape Benjamin D. Esham / Wikimedia Commons}}
    \end{column}
  \end{columns}
\end{frame}


\begin{frame}
  \frametitle{Dining Philosophers (and II)}

  \begin{columns}
    \begin{column}{.6\textwidth}
       \begin{itemize}
         \item Fork is a \alert{shared resource}
           \begin{itemize}
             \item[] Two actions: \alert{get}, \alert{put}
           \end{itemize}
         \item Before eating, a philosopher must first
           {\only<2-3>{\color{red}\bf}get} his
           {\only<2>{\color{red}\bf}right} and then {\only<3>{\color{red}\bf}left}
            forks.
          \item Once finished, the philosopher {\only<4>{\color{red}\bf}puts back} both forks.
       \end{itemize}
    \end{column}

    \begin{column}{.39\textwidth}
      \only<1>{\includegraphics[width=.9\textwidth]{figures/dining_philosophers.png}}%
      \only<2>{\includegraphics[width=.9\textwidth]{figures/dining_philosophers_right.png}}%
      \only<3>{\includegraphics[width=.9\textwidth]{figures/dining_philosophers_both.png}}%
      \only<4>{\includegraphics[width=.9\textwidth]{figures/dining_philosophers_stop.png}}
      \hspace*{15pt}\hbox{\tiny Source:\thinspace{\itshape Benjamin D. Esham / Wikimedia Commons}}
    \end{column}
  \end{columns}
\end{frame}

\begin{frame}[t, fragile]
  \frametitle{Dining Philosophers in Java -- Forks}

  Forks are passive entities -- implemented as monitors.
  \vspace{.3cm}

  \begin{tcolorbox}[colback=white,colframe=DarkGreen]
  \begin{minted}[fontsize=\tiny]{java}
class Fork {
  private boolean taken=false;
  private int identity;

  public Fork(int id) { identity = id; }

  public synchronized void put() {
    taken=false;
    System.out.println("Fork " + identity + " is free");
    notify();
  }

  public synchronized void get() throws java.lang.InterruptedException {
    while (taken) wait();
    taken=true;
    System.out.println("Fork " + identity + " is taken");
  }
}
  \end{minted}
  \end{tcolorbox}
\end{frame}

\begin{frame}[t,fragile]
  \frametitle{Dining Philosophers in Java -- Philosophers}

  Philosophers are active entities -- implemented as threads.
  \vspace{.3cm}

  \begin{tcolorbox}[colback=white,colframe=DarkGreen]
  \begin{minted}[fontsize=\tiny]{java}
class Philosopher extends Thread {
  ...
  public void run() {
    try {
      while (true) {
        System.out.println("Philosopher " + identity + " is thinking...");
        sleep(thinkTime());
        System.out.println("Philosopher " + identity + " is hungry!");
        right.get();
        System.out.println("Philosopher " + identity + " got right fork.");
        sleep(500);
        left.get();
        System.out.println("Philosopher " + identity + " got left fork.");
        System.out.println("Philosopher " + identity + " is eating.");
        sleep(eatTime());
        right.put();
        left.put();
      }
    } catch (java.lang.InterruptedException e) {}
  }
}
  \end{minted}
  \end{tcolorbox}
\end{frame}

\begin{frame}
  \frametitle{Dining Philosophers in Java}
  \centering
  \begin{tcolorbox}[colback=white,colframe=DarkGreen]
    \centering
    \vspace{1cm}
    \Huge DEMO!
    \vspace{1cm}
  \end{tcolorbox}
\end{frame}

\begin{frame}
  \frametitle{How Can the Philosophers Deadlock?}

  \begin{columns}
    \begin{column}{.6\textwidth}
      \uncover<2->{
      Steps:
      \begin{enumerate}
          \item Plato gets right fork.
          \item Konfunzius gets right fork.
          \item Socrates gets right fork.
          \item Voltaire gets right fork.
          \item Descartes gets right fork.
      \end{enumerate}}\uncover<3->{
  \begin{tcolorbox}[colback=white,colframe=DarkRed]
    They are all in a wait-cycle!
  \end{tcolorbox}
      }
    \end{column}
    \begin{column}{.39\textwidth}
      \uncover<3>{\includegraphics[width=.9\textwidth]{figures/dining_philosophers_deadlock.png}
      \hspace*{15pt}\hbox{\tiny Source:\thinspace{\itshape Benjamin D. Esham / Wikimedia Commons}}
      }
    \end{column}
  \end{columns}
\end{frame}


\begin{frame}
  \frametitle{Deadlock-Free Philosophers}
  \begin{itemize}
    \item[] Deadlock can be avoided by ensuring that a wait-for cycle cannot
      exist. \alert{How?}
      \vspace{1cm}
    \item[] Introduce some asymetry in the way the philosophers lock the forks.
    \item[] Number each fork from 1 to 5. Always lock first the
      fork with \emph{lower} identifier.
      \vspace{1cm}
    \item[] Exercise: \alert{can you think of other strategies?}
  \end{itemize}
\end{frame}

\begin{frame}
  \frametitle{Summary}

  We have seen:
      \begin{itemize}
        \item[{\color{DarkGreen}$\checkmark$}] The concept of deadlock
        \item[{\color{DarkGreen}$\checkmark$}] The Dining Philosophers Problem
        \item[{\color{DarkGreen}$\checkmark$}] Four necessary and sufficient conditions for deadlocks
        \item[{\color{DarkGreen}$\checkmark$}] Deadlocks in Java
      \end{itemize}

      \vspace{.7cm}
  \begin{tcolorbox}[colback=white,colframe=DarkRed]
    \begin{itemize}
      \item \alert{Aim:} avoding deadlocks!
      \item Avoiding deadlocks is hard, and an active field of research.
    \end{itemize}
  \end{tcolorbox}
\end{frame}

\section{Smol Zooid}

\subsection{Processes}

\begin{frame}
    \frametitle{Goals}
    \begin{sticky}
    \begin{itemize}
    \item Certifying processes
    \item Extracting runnable code
    \item Avoiding complex formalisations of binders, whenever possible
    \end{itemize}
    \end{sticky}
\end{frame}

\begin{frame}
    \frametitle{Smol Zooid}
    \begin{greenbox}{}
    \begin{itemize}
    \item We combine \textbf{shallow/deep embeddings} of binders
    \begin{itemize}
    \item We use DeBruijn indices for the deeply embedded binders
    \end{itemize}
    \item SZooid constructs are \textbf{well-typed by construction}
    \item We leverage \textbf{Coq code extraction} mechanism
    \item For simplicity, SZooid does not cover \emph{choices}
    \end{itemize}
    \end{greenbox}
\end{frame}

\begin{frame}[fragile]
    \frametitle{Core Processes}
    \begin{minted}{coq}
Inductive proc :=
| Inact
| Rec (e : proc)
| Jump (X : nat)
| Send (p : participant) {T : type}
     (x : interp_type T) (k : proc)
| Recv (p : participant) {T : type}
    (k : interp_type T -> proc)

| ReadIO {T : type} (k : interp_type T -> proc)
| WriteIO {T : type} (x : interp_type T) (k : proc)
.
    \end{minted}    
\end{frame}

\begin{frame}[fragile]
    \frametitle{Payload Types}
    We need to define a type for payload types:
    \begin{itemize}
    \item We need a decidable equality on payload types
    \item We need a decidable equality on payload values
    \end{itemize}
\vspace{1cm}
    \begin{minted}{coq}
Inductive type := Nat | Bool | ...
Definition interp_type : type -> Type := ...
    \end{minted}
\end{frame}

\subsection{Traces}

\begin{frame}[fragile]
    \frametitle{Semantics: Recursion Variables and I/O}
\begin{itemize}
    \item We need some functions to handle recursion variables
    \item \texttt{p\_unroll} exposes the first communication action in a process by
    ``running'' any I/O action, and unfolding recursion
    \item I/O actions are handled using \textbf{parameters} that will be instantiated
only in the OCaml side, when doing code extraction
\end{itemize}
\vspace{1cm}
    \begin{minted}{coq}
Variable readIO : forall {T}, unit -> interp_type T.
Variable writeIO : forall {T}, interp_type T -> unit.

Definition p_subst : nat -> proc -> proc := ...
Definition p_unroll : proc -> proc := ...
    \end{minted}
\end{frame}

\begin{frame}[fragile]
    \frametitle{Semantics: events}
    The semantics is an LTS: 
    \begin{itemize}
    \item the labels are the \textbf{communication events}
    \item it is parameterised by a \textbf{payload interpretation function}
    \end{itemize}
\vspace{.5cm}
    \begin{minted}{coq}
Inductive action := a_send | a_recv.
Record event interp_payload :=
  { action_type  : action;
    subj         : participant;
    party        : participant;
    payload_type : type;
    payload      : interp_payload payload_type
  }.
    \end{minted}
\end{frame}

\begin{frame}[fragile]
    \frametitle{Semantics: step}
The step of the LTS is defined as a \textbf{function}:
\vspace{.5cm}
    \begin{minted}{coq}
Definition step' e E :=
  match e with
  | Send p T x k =>
    if (action_type E == a_send) && (party E == p) &&
       (eq_payload (payload E) x)
    then Some k else None
  | Recv p T k => ...  | _ => None
  end.
Definition step e := step' (p_unroll e).
    \end{minted}
\end{frame}

\begin{frame}[fragile]
    \frametitle{Semantics: traces}
Traces are generated coinductively, by the greatest fixpoint of:
\vspace{.5cm}
    \begin{minted}{coq}
Definition R_trace := rt_trace -> proc -> Prop.
Inductive proc_lts_ p (G : R_trace) : R_trace :=
  | p_end :
      @proc_lts_ p G tr_end Inact
  | p_next E TRC e e' :
      subj E == p ->
      step e E = Some e' ->
      G TRC e' ->
      @proc_lts_ p G (tr_next E TRC) e.
    \end{minted}
\end{frame}

\subsection{Local Types}

\begin{frame}[fragile]
    \frametitle{Local Types}
    We introduce a typing discipline that associates processes with \textbf{local types}, that 
characterise their communication behaviour:

    \begin{minted}{coq}
Inductive lty :=
  | l_end
  | l_jump (X : nat)
  | l_rec (k : lty)
  | l_send (p : participant) (T : type) (l : lty)
  | l_recv (p : participant) (T : type) (l : lty)
.
    \end{minted}
\end{frame}

\begin{frame}[fragile]
    \frametitle{Type System}
    \begin{minted}{coq}
Inductive of_lty : proc -> lty -> Prop :=
| lt_Send    p T k L x :
    of_lty k L -> of_lty (@Send p T x k) (l_send p T L)
| lt_ReadIO  T k L :
    (forall x, of_lty (k x) L) -> of_lty (@ReadIO T k) L
| ...
.
    \end{minted}
\end{frame}

\begin{frame}[fragile]
    \frametitle{Smol Zooid: Smart Constructors (I)}
    \begin{sticky}
\vspace{-.5cm}
    \begin{itemize}
        \item It would be tedious to type up both a local type and a process
        \item Users would need to provide a proof that processes are well-typed
    \end{itemize}
    \end{sticky}
    \begin{greenbox}{}
        We define \textbf{SZooid} (Smol Zooid), to write
well-typed processes by construction, avoiding repetition.
    \end{greenbox}
\end{frame}

\begin{frame}[fragile]
    \frametitle{Smol Zooid: Smart Constructors (and II)}
\begin{minted}{coq}
Definition SZooid L := { p | of_lty p L}.

Definition z_Send  p T x L (k : SZooid L)
  : SZooid (l_send p T L)
  := exist _ _ (lt_Send p x (proj2_sig k)).
...
\end{minted}
\end{frame}

\begin{frame}[fragile]
    \frametitle{Inferring Local Types}
    SZooid constructs fully determine their types from their inputs, so we can ask Coq to \emph{infer} local types
associated with SZooid terms:
\vspace{1cm}
\begin{minted}{coq}
Definition AZooid := { L & SZooid L }.
\end{minted}
\end{frame}

\subsection{Preservation}

\begin{frame}[fragile]
\frametitle{Preservation}
\begin{minted}{coq}
Theorem preservation (e : proc) (L : lty) 
    (H : of_lty e L) (E : rt_event) :
  forall e',
    step e E = Some e' ->
      exists L', lstep L (ev_erase E) = Some L' /\
                 of_lty e' L'.
\end{minted}
\end{frame}

\subsection{Extraction}

\begin{frame}[fragile]
\frametitle{Extraction}
\begin{itemize}
\item We recursively traverse processes, and replace constructors to function calls in an \textbf{ambient monad} that specifies \textbf{how} to execute the programs
\item The ambient monad needs to be implemented in OCaml
\item Processes are extracted using Higher-Order modules, so it is straightforward to change the underlying
transport
\end{itemize}
\end{frame}

\begin{frame}[fragile]
\frametitle{Process Monad (I)}
\begin{minted}{coq}
Module Type ProcessMonad.
  Parameter t : Type -> Type.

  Parameter run : forall {A}, t A -> A.
  Parameter bind 
    : forall {T1 T2}, t T1 -> (T1 -> t T2) -> t T2.
  Parameter pure : forall {T1}, T1 -> t T1.
\end{minted}
\end{frame}

\begin{frame}[fragile]
\frametitle{Process Monad (and II)}
\begin{minted}{coq}
  Parameter send 
    : forall T, participant -> interp_type T -> t unit.
  Parameter recv
    : forall T, participant -> t (interp_type T).

  Parameter loop 
    : forall {T1}, nat -> (unit -> t T1) -> t T1.
  Parameter set_current: nat -> t unit.
End ProcessMonad.
\end{minted}
\end{frame}

\begin{frame}[fragile]
\frametitle{Example Extraction}
\begin{minted}{coq}
Module  ALICE (MP : ProcessMonad) : PROCESS_FUNCTOR(MP).
  Module PE := ProcExtraction(MP).
  Module PM := MP.
  Definition proc :=
    Eval compute in PE.extract_proc 0 alice.
End ALICE.

Extraction ALICE.
\end{minted}
\end{frame}

\begin{frame}[fragile]
\frametitle{Summary}
\begin{sticky}
\begin{itemize}
\item We have seen how to encode a small calculus of Multiparty Processes, with a basic type system
\item Choices complicate the formalisation substatially
\item \textbf{Next: how do we relate traces of individual processes to a larger system?}
\end{itemize}
\end{sticky}
\end{frame}

%\section{References}
%\bibliographystyle{apalike}
%\bibliography{bibliography}
\end{document}
