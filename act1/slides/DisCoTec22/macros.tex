% Required packages

% \usepackage{proof}
\usepackage{mathpartir}
% \usepackage[svgnames]{xcolor}

% symbols
\usepackage{amsmath}
\usepackage{amssymb}
% \usepackage{dsfont}
% \usepackage{stmaryrd}


% \usepackage[normalem]{ulem} % for strikethrough with \sout{}
\usepackage{cancel}

% useful packages

\usepackage{xspace}
\usepackage{fancyvrb}
\VerbatimFootnotes % to enable the use of verbatim in footnotes, it may conflict with some packages

% syntax highlighting

% \usepackage[skins,listings]{tcolorbox}
\usepackage{listings}
\definecolor{dkblue}{rgb}{0,0.1,0.5}
\definecolor{lightblue}{rgb}{0,0.5,0.5}
\definecolor{dkgreen}{rgb}{0,0.4,0}
\definecolor{dk2green}{rgb}{0.4,0,0}
\definecolor{dkviolet}{rgb}{0.6,0,0.8}
\definecolor{mantra}{rgb}{0.2,0.6,0.2}
\definecolor{gotcha}{rgb}{0.8,0.2,0}
\definecolor{ocre}{RGB}{243,102,25} %borrowed to Orange Book

\def\lstlanguagefiles{defManSSR.tex}
\lstset{language=SSR}

%%% Some macros

% Miscelaneous typesetting
\newcommand{\code}[1]{\texttt{#1}} % typeset with mono spaced font
\newcommand{\rulename}[1]{\DefTirName{#1}}
\newcommand{\fnm}[1]{\ensuremath{\text{#1}}} % the name of a function
\newcommand{\dbj}{de Bruijn\xspace}

%% Color management

% \newcommand{\NOCOLOR} % define to remove color from the syntax

\ifdefined\NOCOLOR
  \newcommand{\withcolor}[2]{#2} % no color selsected
\else
  % sets and restores the color
  \newcommand{\withcolor}[2]{\colorlet{currbkp}{.}\color{#1}{#2}\color{currbkp}}
\fi

% to cancel terms in red instead of black
\ifdefined\NOCOLOR
\else
  \renewcommand{\CancelColor}{\color{red}}
\fi

% Macros for BNF grammars

\newcommand{\bnfas}{\mathrel{::=}}
\newcommand{\bnfalt}{\mathrel{\mid}}

% Optional color definitions

% no color
% \newcommand{\colorch}{black} % color for channel vars
% \newcommand{\colorex}{black} % color for expresion vars
% \newcommand{\colorse}{black} % color for session vars
% \newcommand{\colorlbl}{black} % color for labels
% \newcommand{\colorproc}{black} % color for processes
% \newcommand{\colorexp}{black} % color for expressions
% \newcommand{\colorte}{black} % color for the types of expressions
% \newcommand{\colortp}{black} % color for the types of processes

% some colors
\newcommand{\colorch}{DarkGreen} % color for channel vars
\newcommand{\colorex}{Tomato} % color for expresion vars
\newcommand{\colorse}{Teal} % color for session vars
\newcommand{\colorlbl}{Indigo} % color for labels
\newcommand{\colorproc}{Maroon} % color for processes
\newcommand{\colorexp}{Tomato} % color for expressions
\newcommand{\colorte}{DarkOrchid} % color for the types of expressions
\newcommand{\colortp}{NavyBlue} % color for the types of processes

%% Macros for kinds of vars
% these are to control the rendering of the different kinds of variables
\newcommand{\vch}[1]{\withcolor{\colorch}{#1}} % variables for channels
\newcommand{\vex}[1]{\withcolor{\colorex}{#1}} % variables for expressions
\newcommand{\vse}[1]{\withcolor{\colorse}{#1}} % variables for sessions

%% Macros for labels

\newcommand{\dlbl}[1]{\withcolor{\colorlbl}{#1}} % just to define possible use of colours
\newcommand{\lblleft}{\dlbl{\code{l}}}
\newcommand{\lblright}{\dlbl{\code{r}}}
% \newcommand{\lblleft}{\dlbl{\code{left}}}
% \newcommand{\lblright}{\dlbl{\code{right}}}

%% Macros for processes

\newcommand{\openop}[2]{\ensuremath{{#1}^{#2}}} % term name
\newcommand{\closeop}[2]{^{\backslash #2}\ensuremath{{#1}}} % term name

\newcommand{\dproc}[1]{\withcolor{\colorproc}{#1}} % just to define possible use of colours

\newcommand{\preq}[3]{\dproc{\code{request}\ \vse{#1}\,(\vch{#2}). \dproc{#3}}}
\newcommand{\pacc}[3]{\dproc{\code{accept}\ \vse{#1}\,(\vch{#2}). \dproc{#3}}}
\newcommand{\psend}[3]{\dproc{\vch{#1}\,![\dexp{#2}];\,\dproc{#3}}}
\newcommand{\precv}[3]{\dproc{\vch{#1}\,?(\vex{#2}). \dproc{#3}}}
\newcommand{\psel}[3]{\dproc{\vch{#1}\triangleleft\dlbl{#2};\dproc{#3}}}
\newcommand{\pbran}[3]
  {\dproc{\vch{#1}\triangleright\{\lblleft:\dproc{#2}[\!]\lblright:\dproc{#3}\}}}
\newcommand{\pthrow}[3]{\dproc{\code{throw}\ \vch{#1}\,[\vch{#2}];\dproc{#3}}}
\newcommand{\pcatch}[3]{\dproc{\code{catch}\ \vch{#1}\,(\vch{#2}). \dproc{#3}}}
\newcommand{\pif}[3]
  {\dproc{\code{if}\ \dexp{#1}\ \code{then}\ \dproc{#2}\ \code{else}\ \dproc{#3}}}
\newcommand{\ppar}[2]{\dproc{#1}\mathrel{|}\dproc{#2}}
\newcommand{\pinact}{\dproc{\code{inact}}}
\newcommand{\phide}[3]{\dproc{\nu_{#1}\,({#2}).\dproc{#3}}}
\newcommand{\hideboth}{\{n,c\}}
\newcommand{\phidenm}[2]{\phide{n} {\vse{#1}} {#2}}
\newcommand{\phidech}[2]{\phide{c} {\vch{#1}} {#2}}
\newcommand{\pnu}[1]{\ensuremath{{\dproc{\nu.{#1}}}}}

%% Macros for expressions

\newcommand{\dexp}[1]{\withcolor{\colorexp}{#1}} % just to define possible use of colours

\newcommand{\etrue}{\dexp{\code{true}}}
\newcommand{\efalse}{\dexp{\code{false}}}
\newcommand{\elam}[1]{\dexp{\lambda.}\,#1}
\newcommand{\eapp}{\,}
\newcommand{\ezero}{\dexp{0}}
\newcommand{\eone}{\dexp{1}}
\newcommand{\etwo}{\dexp{2}}

\newcommand{\ett}{\ensuremath{\dexp{\texttt{tt}}}}
\newcommand{\eff}{\ensuremath{\dexp{\texttt{ff}}}}
\newcommand{\eunit}{\ensuremath{\dexp{()}}}

%% Macros for types

%\newcommand{\dual}[1]{\bar{#1}}
\newcommand{\dual}[1]{\overline{#1}}

% expressions
\newcommand{\dte}[1]{\withcolor{\colorte}{#1}} % just to define possible use of colours

\newcommand{\tbool}{\dte{\code{bool}}}
\newcommand{\tendp}[1]{\dte{\langle\dtp{#1},\dtp{\dual{#1}}\rangle}}

% processes
\newcommand{\dtp}[1]{\withcolor{\colortp}{#1}} % just to define possible use of colours

\newcommand{\tsende}[2]{\dtp{![\dte{#1}];{#2}}}
\newcommand{\tsendp}[2]{\dtp{![{#1}];{#2}}}
\newcommand{\trecve}[2]{\dtp{?[\dte{#1}];{#2}}}
\newcommand{\trecvp}[2]{\dtp{?[{#1}];{#2}}}
\newcommand{\toffer}[2]{\dtp{\&\{\lblleft:{#1}, \lblright:{#2}\}}}
\newcommand{\ttake}[2]{\dtp{\oplus\{\lblleft:{#1}, \lblright:{#2}\}}}
\newcommand{\tend}{\dtp{\code{end}}}
\newcommand{\tbot}{\dtp{\bot}}

%% Macros for judgments and such

\newcommand{\scon}{\equiv}
\newcommand{\sconalpha}{\equiv_\alpha}
\newcommand{\freenames}[1]{\fnm{fn}({#1})}
\newcommand{\stepsto}{\longrightarrow} % process steps to
\newcommand{\evalsto}{\downarrow} % expression evaluates to
\newcommand{\subst}[2]{[{#1}/{#2}]} % substitution
\newcommand{\esubst}[2]{[\dexp{#1}/\dexp{#2}]} % substitution for expressions

\newcommand{\oft}{\mathrel{:}} % is of type
\newcommand{\oftj}{\mathrel{\triangleright}} % is of type in a process judgment
\newcommand{\oftc}{\mathrel{:}} % is of type in a process judgment
\newcommand{\ofte}[3]{\dexp{#1}\vdash\dexp{#2}\oftc\dtp{#3}}
\newcommand{\oftp}[3]{\dexp{#1}\vdash\dproc{#2}\oftj\dproc{#3}}

\newcommand{\completed}[1]{\dproc{#1}\ \text{completed}}

% Macros for contexts and typings

\newcommand{\sorting}[1]{\dexp{#1}}
\newcommand{\typing}[1]{\dproc{#1}}
\newcommand{\emptyctx}{\ensuremath{\cdot}}
\newcommand{\typc}[3]{\dproc{#1},\vch{#2}\oft\dtp{#3}} % add to a typing
\newcommand{\ctxc}[3]{\dexp{#1},\vex{#2}\oft\dte{#3}} % add to a context

\newcommand{\typl}[3]{\dexp{\Gamma}(\vex{#2})=\dte{#3}} % lookup typ
\newcommand{\ctxl}[3]{\dexp{#1}(\vex{#2})=\dte{#3}} % lookup context

\newcommand{\join}[2]{\dproc{#1}\circ\dproc{#2}}
\newcommand{\comp}[2]{\dproc{#1}\asymp\dproc{#2}}

\newcommand{\dom}[1]{\ensuremath{\fnm{dom}({#1})}}
\newcommand{\doms}[1]{\ensuremath{\fnm{dom}(\sorting{#1})}} % domain of a sorting
\newcommand{\domt}[1]{\ensuremath{\fnm{dom}(\typing{#1})}} % domain of a typing
\newcommand{\compatible}[2]{\ensuremath{\typing{#1}\mathrel{\asymp}\typing{#2}}}
\newcommand{\compose}[2]{\ensuremath{\typing{#1}\mathrel{\circ}\typing{#2}}}

\newcommand{\fv}[1]{\ensuremath{\fnm{fv}({#1})}} % free variables
\newcommand{\fvp}[1]{\fv{\dproc{#1}}} % free variables in a proc


%% Macros for LN things

\newcommand{\atoms}{\ensuremath{\mathbb{A}}}
\newcommand{\atomset}[1]{\atoms_{\code{#1}}}
\newcommand{\atomsca}{\atomset{CA}}
\newcommand{\atomsna}{\atomset{NA}}
\newcommand{\atomska}{\atomset{KA}}
\newcommand{\atomsea}{\atomset{EA}}

% open/close process with expression
\newcommand{\oppe}[2]{\dproc{#1}^{\dexp{#2}}}
\newcommand{\clpe}[2]{^{\backslash\dexp{#2}}\dproc{#1}}

% open/close process with channel
\newcommand{\oppk}[2]{\dproc{#1}^{\vch{#2}}}
\newcommand{\clpk}[2]{^{\backslash\vse{#2}}\dproc{#1}}

% open/close process with session (name)
\newcommand{\oppn}[2]{\dproc{#1}^{\vse{#2}}}
\newcommand{\clpn}[2]{^{\backslash\vse{#2}}\dproc{#1}}

% co finite quantifications

\newcommand{\cofin}[2]{\forall {#1}\notin {#2},\ }
\newcommand{\cofine}[2]{\cofin{\vex{#1}} {#2}}
\newcommand{\cofink}[2]{\cofin{\vch{#1}} {#2}}
\newcommand{\cofinn}[2]{\cofin{\vse{#1}} {#2}}

% predicates

\newcommand{\pred}[2]{\ensuremath{\dproc{#1}\mathrel{\to}\dproc{#2}}} % process reduction

\newcommand{\lc}[1]{\fnm{lc}\ {#1}} % locally closed
\newcommand{\lcp}[1]{\lc{\dproc{#1}}}
\newcommand{\lce}[1]{\lc{\dexp{#1}}}

\newcommand{\body}[1]{\fnm{body}\ {#1}} % at most one free index (0)
\newcommand{\bodyp}[1]{\body{\dproc{#1}}}
\newcommand{\bodye}[1]{\body{\dexp{#1}}}

% some names

\newcommand{\nuscr}{$\nu$-Scr}

%%% Local Variables:
%%% mode: latex
%%% TeX-master: "draft"
%%% End:

