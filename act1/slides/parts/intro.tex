\section{Introduction}

% \endinput
\begin{frame}
  \vfill
  \centering
  %\begin{beamercolorbox}[sep=8pt,center,shadow=true,rounded=true]{block}
  \begin{sticky}
    \usebeamerfont{title}{\underline{Part I:}} \\
    {\normalfont\Large Locally Nameless at a glance} \par%
  \end{sticky}
  %\end{beamercolorbox}
  \vfill
\end{frame}

\newcommand{\bvar}[1]{\ensuremath{#1}}
\newcommand{\fvar}[1]{\ensuremath{#1}}
\newcommand{\lam}[1]{\ensuremath{\lambda\mathrel{.}{#1}}}
\newcommand{\app}[2]{\ensuremath{{#1}\,{#2}}}
\newcommand{\base}[0]{\ensuremath{\texttt{base}}}
\newcommand{\arr}[2]{\ensuremath{{#1}\to{#2}}}

\newcommand{\loft}[3]{\ensuremath{{#1}\mathrel{\vdash}{#2}\mathrel{:}{#3}}}

\begin{frame}
  \frametitle{The Simply Typed $\lambda$-calculus -- STLC}
  \begin{block}{Syntax:}
    \begin{displaymath}
      \begin{array}{rrcl}
        \text{Terms} & M, N & \mathrel{::=} & \bvar n \mid \fvar x \mid \lam M \mid \app M N\\
        \text{Types} & S, T & \mathrel{::=} & \base \mid \arr S T\\
      \end{array}
    \end{displaymath}

  \end{block}
  \begin{block}{Locally Nameless Operations:}
    \begin{displaymath}
      \begin{array}{cc}
        \open &
      \end{array}
    \end{displaymath}
  \end{block}
  \begin{block}{Typing:}
      \begin{mathpar}\centering
        \inferrule{\Gamma(\fvar x) = T}{\loft\Gamma{\fvar x}{T}} \and

        \inferrule{\loft\Gamma M {\arr S T} \\ \loft\Gamma N S}
        {\loft\Gamma{\app M N} T} \and

        \inferrule{\forall x \notin L \\ \loft{\Gamma, x\oft S}{M}T}
        {\loft\Gamma{\lam M}{\arr S T}}
      \end{mathpar}
    \end{block}

\end{frame}

\begin{frame}
  \frametitle{About Today's Lecture}

  \begin{block}{So far in this course:}
      \begin{itemize}
        \item[{\color{DarkGreen}$\checkmark$}] Java Thread creation
        \item[{\color{DarkGreen}$\checkmark$}] Synchronisation: Java \mintinline{java}{synchronized}, monitors
      \end{itemize}
  \end{block}

  \begin{block}{Today}
      \begin{itemize}
        \item[{\color{DarkBlue}$\blacksquare$}] Deadlocks
        \item[{\color{DarkBlue}$\blacksquare$}] The Dining Philosophers Problem
      \end{itemize}
  \end{block}

  \begin{block}{\color{DarkGreen}Intended Learning Outcomes}
      \begin{itemize}
        \item[{\color{DarkGreen}$\blacksquare$}] The concept of \textbf{deadlock}
        \item[{\color{DarkGreen}$\blacksquare$}] Four necessary and sufficient conditions for deadlocks
        \item[{\color{DarkGreen}$\blacksquare$}] Identifying deadlocks in practice: blocked threads
      \end{itemize}
  \end{block}
\end{frame}

\begin{frame}
  \frametitle{Deadlock}
  \begin{tcolorbox}[colback=white,colframe=DarkGreen]
    \begin{tabularx}{\linewidth}{l l}
      \textbf{Concept:} & \alert{deadlock}: no further progress possible
      \\
      & there are four necessary \& sufficient conditions
      \\
      \textbf{In Practice:} & blocked threads
    \end{tabularx}
  \end{tcolorbox}

  \begin{minipage}{.4\textwidth}
    \hspace{10cm}
  \end{minipage}
  \begin{minipage}{.54\textwidth}
  \begin{alertblock}{Aim}
    \textbf{Deadlock-avoidance}: design and implement systems where deadlock cannot
    occur.
  \end{alertblock}
  \end{minipage}
\end{frame}

\begin{frame}
  \frametitle{Four necessary \& sufficient conditions for deadlocks}
  \begin{tcolorbox}[colback=white,colframe=DarkGreen]
    \begin{enumerate}
      \item \alert<2>{Serially reusable resources}
        \uncover<2->{
          \begin{itemize}
            \item[] the processes involved share resources which they use under mutual exclusion
          \end{itemize}
        }
      \item \alert<3>{Incremental acquisition}
        \uncover<3->{
          \begin{itemize}
            \item[] processes hold on to resources already allocated to them
              while waiting to acquire additional resources
          \end{itemize}
        }
        \item \alert<4>{No preemption}
        \uncover<4->{
          \begin{itemize}
            \item[] once acquired by a process, resources cannot be pre-empted
              (forcibly withdrawn) but are only released voluntarily
          \end{itemize}
        }
        \item \alert<5>{Wait-for cycle}
        \uncover<5->{
          \begin{itemize}
            \item[] a circular chain (or cycle) of processes exists such that
              each process holds a resource which its successor in the cycle is
              waiting to acquire
          \end{itemize}
        }
    \end{enumerate}
  \end{tcolorbox}
\end{frame}

\begin{frame}
  \frametitle{Wait-for cycle}

  \begin{tikzpicture}
    \node[draw, ultra thick, circle, minimum width=1cm] at (180:2cm) (A) { A };

    \node[draw, ultra thick, circle, minimum width=1cm] at (60:2cm) (B) { B };

    \node[draw, ultra thick, circle, minimum width=1cm] at (300:2cm) (C) { C };

    \node[draw, ultra thick, circle, minimum width=1cm, right=2cm of B] (Z) { Z };

    \node[left=.7cm of A] {\color{red}holds A, awaits B};
    \node[above=.4cm of B] {\color{red}holds B, awaits C};
    \node[below=.4cm of C] {\color{red}holds C, awaits A};
    \node[below=.4cm of Z] {\color{red}holds Z, awaits B};
    \draw[bend left=45, thick, ->,tips=proper] (A) to (B);
    \draw[bend left=45, thick, ->,tips=proper] (B) to (C);
    \draw[bend left=45, thick, ->,tips=proper] (C) to (A);
    \draw[thick, ->, tips=proper] (Z) -- (B);
  \end{tikzpicture}
\end{frame}

\begin{frame}
  \frametitle{Dining Philosophers (I)}

  \begin{columns}
    \begin{column}{.6\textwidth}
       Five philosophers sit around a circular table.
       \begin{itemize}
           \item Each philosopher spends
       his life alternately \alert{thinking} and \alert{eating}.
       % In the centre of the table is a large bowl of spaghetti.
           \item A philosopher needs \alert{two forks} to start eating.

          \item One fork is placed between each pair of philosophers and they agree that
            each will only use the fork to his immediate right and left.
       \end{itemize}
    \end{column}

    \begin{column}{.39\textwidth}
      \includegraphics[width=.9\textwidth]{figures/dining_philosophers.png}
      \hspace*{15pt}\hbox{\tiny Source:\thinspace{\itshape Benjamin D. Esham / Wikimedia Commons}}
    \end{column}
  \end{columns}
\end{frame}


\begin{frame}
  \frametitle{Dining Philosophers (and II)}

  \begin{columns}
    \begin{column}{.6\textwidth}
       \begin{itemize}
         \item Fork is a \alert{shared resource}
           \begin{itemize}
             \item[] Two actions: \alert{get}, \alert{put}
           \end{itemize}
         \item Before eating, a philosopher must first
           {\only<2-3>{\color{red}\bf}get} his
           {\only<2>{\color{red}\bf}right} and then {\only<3>{\color{red}\bf}left}
            forks.
          \item Once finished, the philosopher {\only<4>{\color{red}\bf}puts back} both forks.
       \end{itemize}
    \end{column}

    \begin{column}{.39\textwidth}
      \only<1>{\includegraphics[width=.9\textwidth]{figures/dining_philosophers.png}}%
      \only<2>{\includegraphics[width=.9\textwidth]{figures/dining_philosophers_right.png}}%
      \only<3>{\includegraphics[width=.9\textwidth]{figures/dining_philosophers_both.png}}%
      \only<4>{\includegraphics[width=.9\textwidth]{figures/dining_philosophers_stop.png}}
      \hspace*{15pt}\hbox{\tiny Source:\thinspace{\itshape Benjamin D. Esham / Wikimedia Commons}}
    \end{column}
  \end{columns}
\end{frame}

\begin{frame}[t, fragile]
  \frametitle{Dining Philosophers in Java -- Forks}

  Forks are passive entities -- implemented as monitors.
  \vspace{.3cm}

  \begin{tcolorbox}[colback=white,colframe=DarkGreen]
  \begin{minted}[fontsize=\tiny]{java}
class Fork {
  private boolean taken=false;
  private int identity;

  public Fork(int id) { identity = id; }

  public synchronized void put() {
    taken=false;
    System.out.println("Fork " + identity + " is free");
    notify();
  }

  public synchronized void get() throws java.lang.InterruptedException {
    while (taken) wait();
    taken=true;
    System.out.println("Fork " + identity + " is taken");
  }
}
  \end{minted}
  \end{tcolorbox}
\end{frame}

\begin{frame}[t,fragile]
  \frametitle{Dining Philosophers in Java -- Philosophers}

  Philosophers are active entities -- implemented as threads.
  \vspace{.3cm}

  \begin{tcolorbox}[colback=white,colframe=DarkGreen]
  \begin{minted}[fontsize=\tiny]{java}
class Philosopher extends Thread {
  ...
  public void run() {
    try {
      while (true) {
        System.out.println("Philosopher " + identity + " is thinking...");
        sleep(thinkTime());
        System.out.println("Philosopher " + identity + " is hungry!");
        right.get();
        System.out.println("Philosopher " + identity + " got right fork.");
        sleep(500);
        left.get();
        System.out.println("Philosopher " + identity + " got left fork.");
        System.out.println("Philosopher " + identity + " is eating.");
        sleep(eatTime());
        right.put();
        left.put();
      }
    } catch (java.lang.InterruptedException e) {}
  }
}
  \end{minted}
  \end{tcolorbox}
\end{frame}

\begin{frame}
  \frametitle{Dining Philosophers in Java}
  \centering
  \begin{tcolorbox}[colback=white,colframe=DarkGreen]
    \centering
    \vspace{1cm}
    \Huge DEMO!
    \vspace{1cm}
  \end{tcolorbox}
\end{frame}

\begin{frame}
  \frametitle{How Can the Philosophers Deadlock?}

  \begin{columns}
    \begin{column}{.6\textwidth}
      \uncover<2->{
      Steps:
      \begin{enumerate}
          \item Plato gets right fork.
          \item Konfunzius gets right fork.
          \item Socrates gets right fork.
          \item Voltaire gets right fork.
          \item Descartes gets right fork.
      \end{enumerate}}\uncover<3->{
  \begin{tcolorbox}[colback=white,colframe=DarkRed]
    They are all in a wait-cycle!
  \end{tcolorbox}
      }
    \end{column}
    \begin{column}{.39\textwidth}
      \uncover<3>{\includegraphics[width=.9\textwidth]{figures/dining_philosophers_deadlock.png}
      \hspace*{15pt}\hbox{\tiny Source:\thinspace{\itshape Benjamin D. Esham / Wikimedia Commons}}
      }
    \end{column}
  \end{columns}
\end{frame}


\begin{frame}
  \frametitle{Deadlock-Free Philosophers}
  \begin{itemize}
    \item[] Deadlock can be avoided by ensuring that a wait-for cycle cannot
      exist. \alert{How?}
      \vspace{1cm}
    \item[] Introduce some asymetry in the way the philosophers lock the forks.
    \item[] Number each fork from 1 to 5. Always lock first the
      fork with \emph{lower} identifier.
      \vspace{1cm}
    \item[] Exercise: \alert{can you think of other strategies?}
  \end{itemize}
\end{frame}

\begin{frame}
  \frametitle{Summary}

  We have seen:
      \begin{itemize}
        \item[{\color{DarkGreen}$\checkmark$}] The concept of deadlock
        \item[{\color{DarkGreen}$\checkmark$}] The Dining Philosophers Problem
        \item[{\color{DarkGreen}$\checkmark$}] Four necessary and sufficient conditions for deadlocks
        \item[{\color{DarkGreen}$\checkmark$}] Deadlocks in Java
      \end{itemize}

      \vspace{.7cm}
  \begin{tcolorbox}[colback=white,colframe=DarkRed]
    \begin{itemize}
      \item \alert{Aim:} avoding deadlocks!
      \item Avoiding deadlocks is hard, and an active field of research.
    \end{itemize}
  \end{tcolorbox}
\end{frame}
