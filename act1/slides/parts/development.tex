
% macros

%% Color management

% \def\NOCOLOR{} % define to remove color from the syntax
\ifdefined\NOCOLOR
  \newcommand{\withcolor}[2]{#2} % no color selsected
\else
  % sets and restores the color
  \newcommand{\withcolor}[2]{\colorlet{currbkp}{.}\color{#1}{#2}\color{currbkp}}
\fi

% % to cancel terms in red instead of black
% \ifdefined\NOCOLOR
% \else
%   \renewcommand{\CancelColor}{\color{red}}
% \fi

% Macros for BNF grammars

\newcommand{\bnfas}{\mathrel{\Coloneqq}}
\newcommand{\bnfalt}{\mathrel{\mid}}

% Optional color definitions

% no color
% \newcommand{\colorch}{black} % colour for channel vars
% \newcommand{\colorex}{black} % colour for expresion vars
% \newcommand{\colorse}{black} % colour for session vars
% \newcommand{\colorlbl}{black} % colour for labels
% \newcommand{\colorproc}{black} % colour for processes
% \newcommand{\colorexp}{black} % colour for expressions
% \newcommand{\colorte}{black} % colour for the types of expressions
% \newcommand{\colorlp}{black} % colour for local types
% \newcommand{\colorgt}{black} % colour for global types

% some colors
\newcommand{\defcl}[1]{\withcolor{black}{#1}} %default colour

% for session types
\newcommand{\colorch}{blue} % colour for channel vars
\newcommand{\colorex}{teal} % colour for expresion vars % Tomato was too bright
\newcommand{\colorproc}{blue} % colour for processes
\newcommand{\colorexp}{teal} % colour for expressions
\newcommand{\colorte}{purple} %{DarkOrchid} is undefined % colour for the types of expressions
\newcommand{\colorst}{Maroon}%{MyDarkOrange} %{NavyBlue} is undefined % colour for session types
% \newcommand{\colorgt}{violet} % colour for global types
% \newcommand{\colorcog}{pine} % colour for coinductive global types
% \newcommand{\colorcol}{dkolive} % colour for coinductive local types
% \newcommand{\colorpre}{coffee} %colour for prefixes

% for the lambda calculus
\newcommand{\colorlctp}{NavyBlue} % colour for the types of terms
\newcommand{\colorlctm}{FireBrick} %for terms
\newcommand{\colorlcvar}{DarkOrchid} % purple for bound variables
\newcommand{\colorrules}{black} % color for rule names

% macros
%% Macros for kinds of vars
% these are to control the rendering of the different kinds of variables
\newcommand{\vch}[1]{\withcolor{\colorch}{#1}} % variables for channels
\newcommand{\vex}[1]{\withcolor{\colorex}{#1}} % variables for expressions

% square brackets in rule names
\let\DefTirNameOld\DefTirName
\renewcommand{\DefTirName}[1]{\withcolor\colorrules{\DefTirNameOld{\footnotesize[{#1}]}}}


\newcommand{\dexp}[1]{\withcolor{\colorexp}{#1}} % just to define possible use o
\newcommand{\dte}[1]{\withcolor{\colorte}{#1}} % just to define possible use of colours

\newcommand{\dproc}[1]{\withcolor{\colorproc}{#1}} % just to define possible use of colours
\newcommand{\dst}[1]{\withcolor{\colorst}{#1}} % just to define possible use of colours

%% Macros for processes

\newcommand{\bvar}[1]{\ensuremath{\dexp{#1}}}
\newcommand{\fvar}[1]{\ensuremath{\dexp{#1}}}
\newcommand{\lam}[1]{\ensuremath{\dexp{\lambda\mathrel{.}{#1}}}}
\newcommand{\app}[2]{\ensuremath{\dexp{#1}\,\dexp{#2}}}
\newcommand{\base}[0]{\ensuremath{\dte{\texttt{base}}}}
\newcommand{\arr}[2]{\ensuremath{\dte{{#1}\to{#2}}}}
\newcommand{\oft}{\ensuremath{\mathrel{:}}} % is of type
\newcommand{\lclosed}[1]{\ensuremath{\texttt{lc}({#1})}}

\newcommand{\loft}[3]{\ensuremath{\vex{#1}\mathrel{\vdash}{\dexp{#2}}\mathrel{:}{\dte{#3}}}}

\newcommand{\openop}[2]{\ensuremath{{#1}^{#2}}} % term name
\newcommand{\closeop}[2]{^{\backslash #2}\ensuremath{{#1}}} % term name

\newcommand{\ett}{\ensuremath{\dexp{\texttt{tt}}}}
\newcommand{\eff}{\ensuremath{\dexp{\texttt{ff}}}}
\newcommand{\eone}{\ensuremath{\dexp{()}}}

\newcommand{\tbool}{\ensuremath{\dte{bool}}}
\newcommand{\tunit}{\ensuremath{\dte{unit}}}

\newcommand{\psend}[3]{\ensuremath{\dproc{{#1}![\dexp{#2}].{#3}}}}
\newcommand{\precv}[3]{\ensuremath{\dproc{{#1}?(\dexp{#2}).{#3}}}}
\newcommand{\pif}[3]{\ensuremath{\dproc{
      \texttt{if}\,\dexp{#1}\mathrel{\texttt{else}}{#2}\mathrel{\texttt{else}}{#3}}}}
\newcommand{\ppar}[2]{\ensuremath{{\dproc{{#1}\mathrel{|}{#2}}}}}
\newcommand{\pnu}[1]{\ensuremath{{\dproc{\nu.{#1}}}}}
\newcommand{\pbang}[1]{\ensuremath{{\dproc{!\,{#1}}}}}
\newcommand{\pinact}[0]{\ensuremath{\dproc{\texttt{inact}}}}

\newcommand{\tin}[2]{\ensuremath\dst{?[\dte{#1}].{#2}}}
\newcommand{\tout}[2]{\ensuremath\dst{![\dte{#1}].{#2}}}
\newcommand{\tend}[0]{\ensuremath\dst{\texttt{end}}}
\newcommand{\tbot}[0]{\ensuremath\dst{\bot}}

\newcommand{\pred}[2]{\ensuremath{\dproc{#1}\mathrel{\to}\dproc{#2}}} % process reduction

\newcommand{\subst}[2]{\ensuremath{[{#1}/{#2}]}}

\newcommand{\eoft}[3]{\ensuremath{\vex{#1}\mathrel{\vdash_e} \dexp{#2}\oft\dte{#3}}}

\newcommand{\poft}[3]{\ensuremath{\dexp{#1}\vdash\dproc{#2}\oft\dproc{#3}}}
\newcommand{\poftG}[2]{\poft{\Gamma}{#1}{#2}}

\newcommand{\completed}[1]{\ensuremath{\texttt{completed}(\vch{#1})}}

\newcommand{\pcompose}[2]{\ensuremath{\vch{#1}\mathrel{\circ}\vch{#2}}}
\newcommand{\pcompatible}[2]{\ensuremath{\vch{#1}\mathrel{\asymp}\vch{#2}}}




\begin{frame}
  \frametitle{The Simply Typed $\lambda$-calculus -- STLC}

%syntax
\begin{displaymath}
  \arraycolsep=1.4pt\def\arraystretch{1.3}
  \begin{array}{rrcl}
    \text{Terms} & \dexp M, \dexp N & \bnfas & \bvar n \bnfalt \fvar x \bnfalt \lam M \bnfalt \app M N\\
    \text{Types} & \dte S, \dte T & \bnfas & \base \bnfalt \arr S T\\
  \end{array}
\end{displaymath}


% LN
\begin{displaymath}
  \arraycolsep=1.4pt\def\arraystretch{1.3}
  \begin{array}{cc}
    \openop {\dexp M} {\vex x} \equiv \{\vex 0 \rightarrow \vex x\} M & \closeop {\dexp M} {\vex x} \equiv \{\vex 0 \leftarrow \vex x\} \dexp M \\
    \multicolumn{2}{c}{\lclosed {\dexp M}}\\
  \end{array}
\end{displaymath}

% typing

\begin{mathpar}
  \inferrule{\vex{\Gamma}(\fvar x) = \dte T}{\loft\Gamma{\fvar x}{T}} \and
  \inferrule{\loft\Gamma M {\arr S T} \\ \loft\Gamma N S}
  {\loft\Gamma{\app M N} T} \and
  \inferrule{\forall \vex x \notin L \\ \loft{\Gamma, x\oft S}{\openop M x} T}
  {\loft\Gamma{\lam M}{\arr S T}}
\end{mathpar}
\end{frame}


\begin{frame}
  \frametitle{The Simply Typed $\lambda$-calculus -- STLC}

\end{frame}


\begin{frame}
  \frametitle{The Simply Typed $\lambda$-calculus -- STLC}

\end{frame}

\begin{frame}
  \frametitle{The Simply Typed $\lambda$-calculus -- STLC}

\end{frame}

\begin{frame}
  \frametitle{The Simply Typed $\lambda$-calculus -- STLC}

\end{frame}

\begin{frame}
  \frametitle{The Simply Typed $\lambda$-calculus -- STLC}

\end{frame}

\begin{frame}
  \frametitle{The Simply Typed $\lambda$-calculus -- STLC}

\end{frame}

\begin{frame}
  \frametitle{The Simply Typed $\lambda$-calculus -- STLC}

\end{frame}

\begin{frame}
  \frametitle{The Simply Typed $\lambda$-calculus -- STLC}

\end{frame}

\begin{frame}
  \frametitle{The Simply Typed $\lambda$-calculus -- STLC}

\end{frame}

\begin{frame}
  \frametitle{The Simply Typed $\lambda$-calculus -- STLC}

\end{frame}

\begin{frame}
  \frametitle{The Simply Typed $\lambda$-calculus -- STLC}

\end{frame}

\begin{frame}
  \frametitle{The Simply Typed $\lambda$-calculus -- STLC}

\end{frame}

\begin{frame}
  \frametitle{The Simply Typed $\lambda$-calculus -- STLC}

\end{frame}

\begin{frame}
  \frametitle{The Simply Typed $\lambda$-calculus -- STLC}

\end{frame}

\begin{frame}
  \frametitle{The Simply Typed $\lambda$-calculus -- STLC}

\end{frame}
